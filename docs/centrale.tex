\documentclass[a4paper,11pt]{article}
\usepackage[utf8]{inputenc}
\usepackage{lmodern}
\usepackage{verbatim}
\usepackage{amsmath}

%opening
\title{}
\author{}

\begin{document}

\maketitle

\begin{abstract}
	Moto in un campo gravitazionale centrale
\end{abstract}

\section{Il problema di Keplero}
Sia $ p $ una particella di massa $ m $ che si muove in un campo gravitazionale prodotto da una massa equivalente $ K $.
Il campo è descritto dall'equazione
\begin{align*}
	U(r) & = - \frac{\alpha}{r} \\
	\alpha & = m K G \\
	G & = 6.67 \times 10^{-11} \frac{m^3}{kg \, s^2} \\
\end{align*}
\begin{equation}
 \label{alpha}	\alpha = m K G \\
\end{equation}

Per il principio di conservazione dell'energia e del momento d'inerzia abbiamo che
\begin{align*}
	E & = \frac{m}{2} \vec R \cdot \vec R + U(r)\\
	M & = mr^2\dot \varphi
\end{align*}
sono costanti.

Se $ E < 0 $ l'orbita descritta è un'ellisse con semiasse asse maggiore e minore rispettivamente $ a $ e $ b $ dati da
\begin{eqnarray}
 \label{a}	a  = \frac{\alpha}{2 |E|} \\
 \label{b}  b  = \frac{M}{\sqrt{2 m |E|}}
\end{eqnarray}

Il tempo d percorrenza dell'intera orbita invece è dato da
\begin{equation}
\label{T}	T = \pi \alpha \sqrt{\frac{m}{2|E|^3}}
\end{equation}

Poniamo allora di conoscere $ a , b, T $ vediamo di trovare le relazioni con le altre costanti del moto.

\section{Calcolo di $K$}

Dalla \eqref{a} ricaviamo

\begin{align*}
	|E| = \frac{\alpha}{2 a}
\end{align*}

mentre dalla \eqref{T} otteniamo
\begin{align*}
	|E|^3 & = \frac{\pi^2}{2 T^2} m \alpha^2 \\
	\frac{\alpha^3}{8 a^3} & = \frac{\pi^2}{2 T^2} m \alpha^2\\
	\alpha & = \frac{4 \pi^2 a^3}{T^2} m
\end{align*}
e dalla \eqref{alpha}
\begin{align*}
	m K G & = \frac{4 \pi^2 a^3}{T^2} m
\end{align*}
da cui
\begin{equation}
\label{K} K = \frac{4 \pi^2 a^3}{G T^2}
\end{equation}

\section{Calcolo di $E$ in funzione di $m$}

Dalla \eqref{alpha} e \eqref{a} otteniamo
\begin{align*}
 E & = - \frac{ K G}{2 a} m
\end{align*}

\begin{equation}
\label{E} E = - \frac{ 2 \pi^2 a^2}{T^2} m
\end{equation}

\section{Calcolo di $M$ in funzione di $m$}

Dalla \eqref{b} otteniamo
\begin{align*}
M^2 & = 2 b^2 |E| m\\
 & = \frac{4 \pi^2 a^2 b^2}{T^2} m^2
\end{align*}

Da cui
\begin{equation}
\label{M} M = 2 \pi \frac{a b}{T} m
\end{equation}

\section{Calcolo di $r_{min}$ e $r_{max}$ }

Una caratteristica geometrica dell'ellisse è che la somma delle distanze dai fuochi di ogni suo punto è costante e uguale all'asse maggiore $ 2 a $.

\begin{align*}
	 2a = |r - f_1| + |r - f_2| 
\end{align*}

Il semi asse maggiore dell'ellisse è
\begin{align*}
	a = \frac{r_{min} + r_{max}}{2} 
\end{align*}
media artimetica.

La distanza dei due fuochi invece è
\begin{align*}
	(f2-f1) = r_{max} - r_{min}
\end{align*}

All'intersezione dell'ellisse con l'asse minore abbiamo

\begin{align*}
	2a & = |r - f_1| + |r - f_2| \\
	 & = 2 |r - f_1| \\
	a & = |r - f_1| \\
	|r - f_1|^2&  = (\frac{|f_2 - f_1|}{2})^2 + b^2 \\
	a^2&  = \frac{(r_{max} - r_{min})^2}{4} + b^2 \\
	r_{max} - r_{min} & = 2\sqrt{(a^2-b^2)}
	r_{max} + r_{min} & = 2a
\end{align*}
da cui
\begin{eqnarray}
	\label{rmax} r_{max} = a + \sqrt{a^2-b^2}\\
	\label{rmin} r_{min} = a - \sqrt{a^2-b^2}
\end{eqnarray}

inversamente abbiamo
\begin{align*}
	b^2 & = a^2 - \frac{(r_{max} - r_{min})^2}{4} \\
	 & = \frac{(r_{max} + r_{min})^2 - (r_{max} -r_{min})^2 }{4} \\
	 & = \frac{(r_{max} + r_{min} + r_{max} -r_{min}) (r_{max} + r_{min} - r_{max} +r_{min})^2 }{4} \\
	 & = \frac{2 r_{max} 2 r_{min} }{4} \\
\end{align*}
da cui
\begin{eqnarray}
\label{b1} b = \sqrt{r_{max} r_{min}}
\end{eqnarray}
media geometrica.

\section{Esempio}

Prendiamo il sistema formato da terra-sole e conoscendo la distanza del perielio e dell'afelio della terra, il periodo di rivoluzione e la massa della terra.
\begin{align*}
	r_{min} & = 147 \times 10^9 m \\
	r_{max} & = 152,1 \times 10^9 m \\
	T & = 365.2425 \cdot 24 \cdot 60 \cdot 60 = 31.56 \times 10^6 s\\
	m & = 5,972 \times 10^{24} kg
\end{align*}

\begin{align*}
	a & = 149,55 \times 10^9 \\
	b & = 149,53 \times 10^9 \\
	K & = \frac{4 \pi^2 a^3}{GT^2}\\
	& = \frac{4 \pi^2 (149,55 \times 10^9)^3}{6.67 \times 10^{-11} (31.56 \times 10^6)^2}\\
	& = \frac{132 \times 10^{33}}{6.644 \times 10^4}\\
	& = 19,868 \times 10^{29}
\end{align*}

Il valore ufficiale della massa solare è
\[
	K = 1,98892 \times 10^{30}
\]


\end{document}
