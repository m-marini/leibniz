\documentclass[a4paper,11pt]{article}
\usepackage[utf8]{inputenc}
\usepackage{lmodern}
\usepackage{verbatim}
\usepackage{amsmath}

%opening
\title{}
\author{}

\begin{document}

\maketitle

\begin{abstract}
	Moto in un campo gravitazionale centrale
\end{abstract}

\section{Il problema di Keplero}
Sia $ p $ una particella di massa $ m $ che si muove in un campo gravitazionale prodotto da una massa equivalente $ K $.
Il campo è descritto dall'equazione
\begin{align*}
	U(r) & = - \frac{\alpha}{r} \\
	\alpha & = m K G \\
	G & = 6.67 \times 10^{-11} \frac{m^3}{kg \, s^2} \\
\end{align*}
\begin{equation}
 \label{alpha}	\alpha & = m K G \\
\end{equation}

Per il principio di conservazione dell'energia e del momento d'inerzia abbiamo che
\begin{align*}
	E & = \frac{m}{2} \vec R \cdot \vec R + U(r)\\
	M & = mr^2\dot \varphi
\end{align*}
sono costanti.

Se $ E < 0 $ l'orbita descritta è un'ellisse con asse maggiore e minore rispettivamente $ a $ e $ b $ dati da
\begin{eqnarray}
 \label{a}	a  = \frac{\alpha}{2 |E|} \\
 \label{b}  b  = \frac{M}{\sqrt{2 m |E|}}
\end{eqnarray}

Il tempo d percorrenza dell'intera orbita invece è dato da
\begin{equation}
\label{T}	T = \pi \alpha \sqrt{\frac{m}{2|E|^3}}
\end{equation}

Poniamo allora di conoscere $ a , b, T $ vediamo di trovare le relazioni con le altre costanti del moto.

\section{Calcolo di K}

Dalla \eqref{a} ricaviamo

\begin{align*}
	|E| = \frac{\alpha}{2 a}
\end{align*}

mentre dalla \eqref{T} otteniamo
\begin{align*}
	|E|^3 & = \frac{\pi^2}{2 T^2} m \alpha^2 \\
	\frac{\alpha^3}{8 a^3} & = \frac{\pi^2}{2 T^2} m \alpha^2\\
	\alpha & = \frac{4 \pi^2 a^3}{T^2} m
\end{align*}
e dalla \eqref{alpha}
\begin{align*}
	m K G & = \frac{4 \pi^2 a^3}{T^2} m
\end{align*}
da cui
\begin{equation}
\label{K} K = \frac{4 \pi^2 a^3}{G T^2}
\end{equation}





\end{document}
